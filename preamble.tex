\newdimen\topmargin
\topmargin=\topspace
\advance \topmargin by \headerdistance

\newcount\mpInst
\mpInst = 0
% Every time new proposition is introduced, we define new picture
% with all its colors and styles, preceded with inclusion of byrne.mp library. 
% It is done with \startMPinclusions in ConTeXt.
% Note that every time new MPinstance is defined .
% Before doing so we clear everything after previous definitions (explained below)
% In case the picture extends the previous picture, CreateNewInstanceForPicturefalse
% can be set temporarily

\def\mpPre{}
\def\mpPost{}

\newif\ifCreateNewInstanceForPicture
\CreateNewInstanceForPicturetrue

\def\textsf#1{\tf\ss #1} % Used within byrne.mp, for compatibility with LaTeX

\def\defineNewPicture{\dodoubleempty\definenewpicture}
\def\definenewpicture[#1][#2]#3{%
	\iffirstargument%
		\global\def\sfA{#1}%
	\else%
		\global\def\sfA{1/3}%
	\fi%
	\ifsecondargument%
		\global\def\sfB{#2}%
	\else%
		\global\def\sfB{defaultScaleFactor}%
	\fi%
	\undefineList%
	\def\undefineList{}%
	\ifCreateNewInstanceForPicture%
		\global\advance \mpInst by 1%
		\global\def\currentInstance{\the\mpInst}%
		\defineMPinstance[\currentInstance]%
	\fi%
	\startreusableMPgraphic{\currentInstance::currentPicture}
		\ifCreateNewInstanceForPicture input byrne; \fi
		\mpPre
		u := 1cm;
		startMainPictureMode;
		projectionAngle := (0, 0, 0);
		linecap := butt;
		scaleFactor := \sfB;
			#3
		scaleFactor := \sfA;
		generateAngleSynonyms;
		generateLineSynonyms;
		stopMainPictureMode;
		\mpPost
	\stopreusableMPgraphic\ignorespaces\setbox0\vbox{\reuseMPgraphic{\currentInstance::currentPicture}}\global\addToUndefineList{#2}
}
	
\def\defineNewPictureBasedOnOld{\dodoubleempty\definenewpicturebasedonold}
\def\definenewpicturebasedonold[#1][#2]#3{%
	\CreateNewInstanceForPicturefalse
	\ifsecondargument%
		\defineNewPicture[#1][#2]{#3}%
	\else%
		\iffirstargument%
			\defineNewPicture[#1]{#3}%
		\else%
			\defineNewPicture{#3}
		\fi%
	\fi%
	\CreateNewInstanceForPicturetrue
	}

% Every time some simple picture is drawn, new macro is defined
% with \markPictAsReady. If macro is defined, predefined picture is used.
% Every time new picture is added, instructions to undefine it are added with
% \addToUndefineList. \undefineList is thus generated command, that, once executed,
% undefines everything that was appended to it with \addToUndefineList.

\def\unmarkPictAsReady#1{\expandafter\let\csname #1\endcsname\undefined}

\def\addToUndefineList#1{\expandafter\def\expandafter\undefineList\expandafter{\undefineList \unmarkPictAsReady{#1}}}

\def\undefineList{}
\addToUndefineList{lastPicture}

% When larger picture is defined, we can visualize its parts with \drawFromCurrentPicture
% First (optional) argument is for picture's vertical alignment relative to the string
% Currently only `middle' and anything other (for `top') is working
% Second (optional) argument is for picture's name for future reuse
% If picture with the same name exists, it's not defined again.
% Third argument is MP code for picture itself.
% To actually define an inline picture \defineFromCurrentPicture is called.
% The latter can also be called separately, which can be convenient
% if one wants to define multiple inline pictures in one place.
% Whenever everything is defined, \drawDefinedPicture macro is called for
\def\drawFromCurrentPicture{\dodoubleempty\drawfromcurrentpicture}
\def\drawfromcurrentpicture[#1][#2]#3{{%
	\iffirstargument%
		\global\def\dfcpl{#1}%
	\else%
		\global\def\dfcpl{middle}%
	\fi%
	\ifsecondargument%
		\expandafter\ifx\csname #2\endcsname\relax%
			\defineFromCurrentPicture{\dfcpl}{#2}{#3}%
			\global\addToUndefineList{#2}%
		\else%
			\global\def\lastPict{#2}%
		\fi%
	\else%
		\defineFromCurrentPicture{\dfcpl}{lastPicture}{#3}%
	\fi%
\csname\lastPict\endcsname}}

\def\defineFromCurrentPicture#1#2#3{%
		\startreusableMPgraphic{\currentInstance::#2}
			startOffspringPictureMode;
			linecap := butt;
			#3
			stopOffspringPictureMode;
		\stopreusableMPgraphic%
\global\def\lastPict{#2}%
\global\expandafter\edef\csname\lastPict\endcsname{{\noexpand\drawDefinedPicture{\expandafter\lastPict}{\expandafter#1}}}}

% \drawDefinedPicture draws defined and named picture, aligned as needed and,
% possibly, with margins modified with \offsetPicture (i. e., if we want picture to act as if it's less tall that it actually is,
% we put some value as its first argument, if we want picture to be less deep, we use second argument, and place picture
% itself as third).

\newdimen\pictOffsetTop
\newdimen\pictOffsetBottom

\pictOffsetTop=0pt
\pictOffsetBottom=0pt

\def\offsetPicture#1#2#3{{\pictOffsetTop=#1\pictOffsetBottom=#2#3\pictOffsetTop=0pt\pictOffsetBottom=0pt}}

\newdimen\midht
\newdimen\middp
\def\drawDefinedPicture#1#2{{%
	\setbox0\vbox{\reuseMPgraphic{\currentInstance::#1}}%
	\ifdim\ht0<\baselineskip\advance\pictOffsetTop by 1pt\advance\pictOffsetBottom by 1pt\fi%
	\setbox0\vbox{\vskip2pt\vskip-\pictOffsetTop\box0\vskip2pt\vskip-\pictOffsetBottom}%
	\def\tmpmiddle{middle}%
	\def\tmpalignment{#2}%
	\ifx\tmpalignment\tmpmiddle%
		\middp=0.5\ht0
		\midht=0.5\ht0
		\advance\middp by -3pt
		\advance\midht by 3pt
	\else%
		\middp=0pt
		\midht=\ht0
		\advance\middp by 3pt
		\advance\midht by -3pt
	\fi%
	\advance\middp by 0.5\pictOffsetTop
	\advance\midht by -0.5\pictOffsetTop
	\advance\middp by -0.5\pictOffsetBottom
	\advance\midht by 0.5\pictOffsetBottom
	\dp0=\middp
	\ht0=\midht
	\nobreak\hskip0pt\nobreak\,\nobreak\box0\,%
}}


% Some convenient shorthand macros for commonly used derivations of definition picture are defined below
\def\drawCurrentPicture{\reuseMPgraphic{\currentInstance::currentPicture}}

\def\drawUnitLine{\dosingleempty\drawunitline}
\def\drawunitline[#1]#2{{%
	\iffirstargument%
		\drawFromCurrentPicture[middle][uline#2#1]{startAutoLabeling;draw byNamedCompoundLine(#1, 0)(#2);stopAutoLabeling;}%
	\else%
		\drawFromCurrentPicture[middle][uline#2]{startAutoLabeling;draw byNamedCompoundLine(1cm, 0)(#2);stopAutoLabeling;}%
	\fi%
}}

\def\drawProportionalLine#1{\drawFromCurrentPicture[middle][pline#1]{startAutoLabeling;draw byNamedCompoundLine(1cm, 1)(#1);stopAutoLabeling;}}

\def\drawSizedLine{\dosingleempty\drawsizedline}
\def\drawsizedline[#1]#2{{%
	\iffirstargument%
		\drawFromCurrentPicture[middle][sline#2#1]{startAutoLabeling;draw byNamedCompoundLine(#1, 2)(#2);stopAutoLabeling;}%
	\else%
		\drawFromCurrentPicture[middle][sline#2]{startAutoLabeling;draw byNamedCompoundLine(2cm, 2)(#2);stopAutoLabeling;}%
	\fi%
}}

\def\drawUnitRay{\dosingleempty\drawunitray}
\def\drawunitray[#1]#2{{%
	\iffirstargument%
		\drawFromCurrentPicture[middle][uray#2#1]{startAutoLabeling;draw byNamedCompoundRay(#1, 0)(#2);stopAutoLabeling;}%
	\else%
		\drawFromCurrentPicture[middle][uray#2]{startAutoLabeling;draw byNamedCompoundRay(1cm, 0)(#2);stopAutoLabeling;}%
	\fi%
}}

\def\drawProportionalRay#1{\drawFromCurrentPicture[middle][rayprop#1]{startAutoLabeling;draw byNamedCompoundRay(1cm, 1)(#1);stopAutoLabeling;}}

\def\drawSizedRay{\dosingleempty\drawsizedray}
\def\drawsizedray[#1]#2{{%
	\iffirstargument%
		\drawFromCurrentPicture[middle][sray#2#1]{startAutoLabeling;draw byNamedCompoundRay(#1, 2)(#2);stopAutoLabeling;}%
	\else%
		\drawFromCurrentPicture[middle][sray#2]{startAutoLabeling;draw byNamedCompoundRay(2cm, 2)(#2);stopAutoLabeling;}%
	\fi%
}}
	
\def\drawUnitIndLine{\dosingleempty\drawunitindline}
\def\drawunitindline[#1]#2{{%
	\iffirstargument%
		\drawFromCurrentPicture[middle][uindline#2#1]{startAutoLabeling;draw byNamedCompoundIndLine(#1, 0)(#2);stopAutoLabeling;}%
	\else%
		\drawFromCurrentPicture[middle][uindline#2]{startAutoLabeling;draw byNamedCompoundIndLine(1cm, 0)(#2);stopAutoLabeling;}%
	\fi%
}}

\def\drawProportionalIndLine#1{\drawFromCurrentPicture[middle][lineindprop#1]{startAutoLabeling;draw byNamedCompoundIndLine(1cm, 1)(#1);stopAutoLabeling;}}

\def\drawSizedIndLine{\dosingleempty\drawsizedindline}
\def\drawsizedindline[#1]#2{{%
	\iffirstargument%
		\drawFromCurrentPicture[middle][lineindsized#2#1]{startAutoLabeling;draw byNamedCompoundIndLine(#1, 2)(#2);stopAutoLabeling;}%
	\else%
		\drawFromCurrentPicture[middle][lineindsized#2]{startAutoLabeling;draw byNamedCompoundIndLine(2cm, 2)(#2);stopAutoLabeling;}%
	\fi%
}}
	
\def\drawRightAngle{%
\drawFromCurrentPicture[middle][onerightangle]{draw rightAngle;}}

\def\drawTwoRightAngles{%
\drawFromCurrentPicture[middle][tworightangles]{draw twoRightAngles;}}
	
\def\drawAngle#1{\drawFromCurrentPicture[middle][angle#1]{startAutoLabeling;draw byNamedAngleWithDummySides(#1);stopAutoLabeling;}}

\def\drawAngleWithSides#1{\drawFromCurrentPicture[middle][anglewithsides#1]{startAutoLabeling;draw byNamedAngleSides(#1)();stopAutoLabeling;}}

\def\drawPolygon{\dodoubleempty\drawpolygon}
\def\drawpolygon[#1][#2]#3{{%
	\iffirstargument%
		\global\def\plal{#1}%
	\else%
		\global\def\plal{middle}%
	\fi%
	\ifsecondargument%
		\drawFromCurrentPicture[\plal][#2]{startAutoLabeling;draw byNamedPolygon(#3);stopAutoLabeling;}%
	\else%
		\drawFromCurrentPicture[\plal][polygon#3]{startAutoLabeling;draw byNamedPolygon(#3);stopAutoLabeling;}%
	\fi%
}}

\def\drawCircle{\dodoubleempty\drawcircle}
\def\drawcircle[#1][#2]#3{{%
	\iffirstargument%
		\global\def\plal{#1}%
	\else%
		\global\def\plal{middle}%
	\fi%
	\ifsecondargument%
		\drawFromCurrentPicture[\plal][circle#3]{
		startAutoLabeling;
		startTempScale(#2);
		draw byNamedCircle(#3);
		stopTempScale;
		stopAutoLabeling;
		}%
	\else%
		\drawFromCurrentPicture[\plal][circle#3]{startAutoLabeling;draw byNamedCircle(#3);stopAutoLabeling;}%
	\fi%
}}

\def\drawArc{\dodoubleempty\drawarc}
\def\drawarc[#1][#2]#3{{%
	\iffirstargument%
		\global\def\plal{#1}%
	\else%
		\global\def\plal{middle}%
	\fi%
	\ifsecondargument%
		\drawFromCurrentPicture[\plal][arc#3]{
		startAutoLabeling;
		startTempScale(#2);
		draw byNamedArc(#3);
		stopTempScale;
		stopAutoLabeling;
		}%
	\else%
		\drawFromCurrentPicture[\plal][arc#3]{startAutoLabeling;draw byNamedArc(#3);stopAutoLabeling;}%
	\fi%
}}

\def\drawLine{\dodoubleempty\drawline}
\def\drawline[#1][#2]#3{{%
	\iffirstargument%
		\global\def\plal{#1}%
	\else%
		\global\def\plal{middle}%
	\fi%
	\ifsecondargument%
		\drawFromCurrentPicture[\plal][#2]{startAutoLabeling;draw byNamedLineSeq(0)(#3);stopAutoLabeling;}%
	\else%
		\drawFromCurrentPicture[\plal][line#3]{startAutoLabeling;draw byNamedLineSeq(0)(#3);stopAutoLabeling;}%
	\fi%
}}
	
\def\drawPointM#1{\drawFromCurrentPicture[middle][pointm#1]{startAutoLabeling; draw byNamedPointMark(#1); stopAutoLabeling;}}

\def\drawPointL{\dodoubleempty\drawpointl}
\def\drawpointl[#1][#2]#3{{%
	\iffirstargument%
		\global\def\plal{#1}%
	\else%
		\global\def\plal{middle}%
	\fi%
	\ifsecondargument%
		\drawFromCurrentPicture[\plal][pointl#3Minus#2]{startAutoLabeling; draw byNamedPointLines(#3,"#2"); stopAutoLabeling;}%
	\else%
		\drawFromCurrentPicture[\plal][pointl#3]{startAutoLabeling; draw byNamedPointLines(#3,""); stopAutoLabeling;}%
	\fi%
}}

\def\drawPoint{\dodoubleempty\drawpoint}
\def\drawpoint[#1][#2]#3{{%
	\iffirstargument%
		\global\def\plal{#1}%
	\else%
		\global\def\plal{middle}%
	\fi%
	\ifsecondargument%
		\drawFromCurrentPicture[\plal][pointl#3Minus#2]{
		startAutoLabeling; draw byNamedPointLines(#3,"#2"); stopAutoLabeling;
		draw byNamedPointMark(#3);
		}%
	\else%
		\drawFromCurrentPicture[\plal][pointl#3]{
		startAutoLabeling; draw byNamedPointLines(#3,""); stopAutoLabeling;
		draw byNamedPointMark(#3);}%
	\fi%
}}

\def\drawMagnitude{\dodoubleempty\drawmagnitude}
\def\drawmagnitude[#1][#2]#3{{%
	\iffirstargument%
		\global\def\plal{#1}%
	\else%
		\global\def\plal{middle}%
	\fi%
	\ifsecondargument%
		\drawFromCurrentPicture[\plal][magnitude#2#3]{startAutoLabeling;draw byNamedMagnitude(#2)(#3);stopAutoLabeling;}%
	\else%
		\drawFromCurrentPicture[\plal][magnitude#3]{startAutoLabeling;draw byNamedMagnitude(0)(#3);stopAutoLabeling;}%
	\fi%
}}