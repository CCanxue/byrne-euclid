% Figure placement it definitions is messy, below we define all the indentation, figure and lettrine placing stuff.
\newdimen\LettrineSize
\newdimen\PlateWidth
\newdimen\PlateHeight
\newdimen\PcI
\newdimen\PcH
\LettrineSize=1.5cm
\PlateHeight=1.5cm
\PlateWidth=7.6cm
\def\LettrineTotalWidth{\dimexpr \LettrineSize + 2mm \relax}
\def\LettrineTotalHeight{\dimexpr \LettrineSize + 2mm \relax}
\def\LettrineTotalHeightMinusLine{\dimexpr \LettrineTotalHeight -0.8\baselineskip}
\def\defaultLettrineLines{4}
\PcI=0.5\textwidth
\newdimen\oddpict
\oddpict=\backspace
\advance \oddpict by \textwidth
\advance \oddpict by -\PcI

\newdimen\evenpict
\evenpict=\cutspace
\advance \evenpict by \PcI

\define\drawCurrentPictureWide{
	\setbox1\vbox{\drawCurrentPicture}
	\newdimen\pictVOffset
	\pictVOffset=\topmargin
	\advance\pictVOffset by \pagetotal
	\PcI=\wd1
	\PcH=\ht1
	\advance\PcH by \pagetotal
	\advance \PcI by 1cm
	\advance \PcI by -\cutspace
	\ifodd\pageno
		\oddpict=\backspace
		\advance \oddpict by \makeupwidth
		\advance \oddpict by -\PcI
		\definelayer[curpic]
			[x=\oddpict,y=\pictVOffset,width=\textwidth,height=\textheight]
	\else
		\evenpict=\cutspace
		\advance \evenpict by \PcI
		\definelayer[curpic]
			[x=0cm,y=\pictVOffset,width=\evenpict,height=\textheight]
	\fi
	\setlayer[curpic][voffset=0.5cm]{\hskip 0.5cm\box1\hskip 0.5cm}
	\setupbackgrounds[page][background=curpic]
}

% Often it's convenient to skip come lines and continue below main figure, 
% leaving all the indentation stuff above
\newdimen\dTB
\def\skipToPictureBottom{
\dTB=\PcH
\advance\dTB by -\pagetotal
\vskip \dTB
\resetII
}

\newdimen\CmI
\newdimen\PgI
\newdimen\NgI
\newcount\Y
\newcount\Yy
\newcount\X
\X=0
\newcount\numOfIndLines

\newif\ifinsideII
\insideIIfalse

\def\resetII{\parshape=0\global\numOfIndLines=0\global\def\par{\endgraf}\everypar=\expandafter{\the\toks0}\global\insideIIfalse}

\newcount\linesofar

\toks0=\expandafter{\the\everypar}

\def\initialIndentation{\dosingleempty\initialindentation}
\def\initialindentation[#1]#2{
	\global\toks0=\expandafter{\the\everypar}
	\iffirstargument
		\def\ltLines{#1}
	\else\ignorespaces
		\def\ltLines{\defaultLettrineLines}
	\fi
	\ifinsideII\resetII\fi
	\global\insideIItrue
	\ \vskip 0pt
 %https://books.google.com/books?id=iDb2BwAAQBAJ&pg=PA107&lpg=PA107&dq=tex+parshape+paragraphs&source=bl&ots=GmFvQnIUr7&sig=oJ-mtVEwOttepWpGi2BABemAMWw&hl=en&sa=X&ved=0ahUKEwjxrbPk-uXKAhVClB4KHSxUBqwQ6AEIQTAF#v=onepage&q=tex%20parshape%20paragraphs&f=false
	\linesofar=0
	\global\numOfIndLines=#2
	\def\par{{\endgraf\global\linesofar=\prevgraf}}
	\everypar={\ifnum\linesofar<\numOfIndLines\prevgraf=\linesofar\else\resetII\fi} %
	\PgI=\textwidth
	\NgI=\textwidth
	\ifodd\pageno
		\advance \PgI by -\PcI
		\CmI=\PgI
		\advance \CmI by -\LettrineTotalWidth
		\def\partindent{0cm \PgI}
		\def\doubleindent{\LettrineTotalWidth \CmI}
	\else
		\advance \PgI by -\PcI
		\CmI=\PcI
		\advance \CmI by \LettrineTotalWidth
		\NgI=\PgI
		\advance \NgI by -\LettrineTotalWidth
		\def\partindent{\PcI \PgI}
		\def\doubleindent{\CmI \NgI}
	\fi
	\def\preinitindent{}
	\Yy=\ltLines
	\advance \Yy by 1 
	\Y=\ltLines
	\loop 
	\expandafter\def\expandafter\preinitindent\expandafter{\preinitindent\doubleindent}
	\advance \Y by -1 
	\ifnum \Y>0
	\repeat
	\def\initindent{\preinitindent}
	\global\X=#2
	\ifnum \X>\Yy
		\advance \X by -\ltLines 
		\loop 
			\expandafter\def\expandafter\initindent\expandafter{\initindent\partindent}
		\advance \X by -1 
		\ifnum \X>1
		\repeat
	\fi
	\parshape=#2
	\initindent
	0cm \textwidth
}

%every lettrine has a counter ltrn#1 that determines which file to pick

\newwrite \lettrineslist
\openout \lettrineslist = lettrines/lettrineslist.txt\relax

\def\writeLettrineSizes{
	\newwrite \lettrinessizes
	\openout \lettrinessizes = lettrines/lettrinessizes.txt\relax
	\write \lettrinessizes{lettrineSize := \the\LettrineSize; plateWidth := \the\PlateWidth; plateHeight := \the\PlateHeight;}
	\closeout \lettrinessizes
}

\writeLettrineSizes % this one can be used is you choose to change lettrine or vignette sizes elsewhere

\def\putLettrine#1{\ignorespaces
\write \lettrineslist{#1}\ignorespaces
\expandafter\ifx\csname ltrn#1\endcsname\relax
	\global\expandafter\newcount\csname ltrn#1\endcsname\relax
	\global\csname ltrn#1\endcsname 0
\else
	\global\expandafter\advance\csname ltrn#1\endcsname by 1
\fi
\doifelsefile{lettrines/#1\the\csname ltrn#1\endcsname.mps}
	{\ifx#1\plateName\ignorespaces
		\externalfigure[lettrines/#1\the\csname ltrn#1\endcsname.mps][width=\PlateWidth]\ignorespaces
	\else\ignorespaces
		\externalfigure[lettrines/#1\the\csname ltrn#1\endcsname.mps][height=\LettrineSize]\ignorespaces
	\fi}
	{\doifelsefile{lettrines/#1.svg}
		{\externalfigure[lettrines/#1.svg][height=\LettrineSize]\ignorespaces}
		{\doifelsefile{lettrines/#1.pdf}
			{\externalfigure[lettrines/#1.pdf][width=\LettrineSize,height=\LettrineSize]\ignorespaces}
			{\ifx#1\plateName\ignorespaces
					\framed[
						width=\PlateWidth,height=\PlateHeight,
						frame=off,
						background=color,
						backgroundcolor=black,
						frameoffset=0cm,
						foregroundcolor=white]{}\ignorespaces
				\else\ignorespaces
					\framed[
						width=\LettrineSize,height=\LettrineSize,
						frame=off,
						background=color,
						backgroundcolor=black,
						foregroundcolor=white]{\vskip-13.5pt\vbox{\tfd #1}}\ignorespaces
				\fi}\ignorespaces}}}
				

% This thing removes space between an initial and the letter after it, if no space is needed
\startluacode
   userdata = userdata or {}
   function userdata.spaceFirst (str)
    if string.sub(str, 0, 1) ==  " " then
      tex.print("".. str)
    else
      tex.print("\\hskip -3pt " .. str)
    end
  end
\stopluacode

% This one is used standalone, f. e. in the Introduction
\def\regularLettrine#1{\placefigure[left,none]{}{\vskip -5pt\hbox{\putLettrine{#1}\hskip-3pt}\vskip -15pt}\noindent}

\def\problem#1#2#3{\ignorespaces
\vskip -\LettrineTotalHeight
\hskip -\LettrineTotalWidth \hskip -\parindent \offset[y=\LettrineTotalHeightMinusLine,x=0pt]{\hbox{\putLettrine{#1}}}{\sc\hskip 5pt\ctxlua{userdata.spaceFirst([==[#2]==])}\ \it #3} \vskip \baselineskip
}

%\setbox0\hbox{\ }
%\newdimen\spaceWd
%\spaceWd = \wd0

\newdimen\vPos
\def\problemNP{\dosingleempty\problemnp}
\def\problemnp[#1]#2#3#4{\ignorespaces
\PcI=0cm\ignorespaces
\vPos=\pagetotal\ignorespaces
\iffirstargument\ignorespaces
	\initialIndentation[#1]{7}\ignorespaces
\else\ignorespaces
	\initialIndentation{7}\ignorespaces
\fi\ignorespaces
\vskip -\LettrineTotalHeight
\hskip -\LettrineTotalWidth \hskip -\parindent \offset[y=\LettrineTotalHeightMinusLine,x=0pt]{\hbox{\putLettrine{#2}}}{\sc\hskip 5pt\ctxlua{userdata.spaceFirst([==[#3]==])}\ \it #4}\par\ignorespaces
\advance\vPos by -\pagetotal
\advance\vPos by 4\baselineskip
\ifnum\vPos>0
\vskip \vPos\ignorespaces
\fi
\vskip 0.5\baselineskip
\resetII}


\def\qedstr{Q. E. D.}
\def\qefstr{Q. E. F.}
\def\qedNB{\hfill\qedstr\vskip 10pt}
\def\qed{\qedNB\vfill\pagebreak}
\def\qefNB{\hfill\qefstr\vskip 10pt}
\def\qef{\qefNB\vfill\pagebreak}

\def\hypstr{hyp.}
\def\conststr{const.}

%ConTeXt specific stuff


\setupcolors[state=start]
\setupinterlinespace[line=3.2ex]
\setupalign[normal,tolerant]

\setupheadertexts[]
\setupheadertexts
	[\hfill\expanded{\uppercase{\fetchmarking[book][][top] \fetchmarking[section][][top]}}\hfill][\pagenumber]
	[\pagenumber][\hfill\expanded{\uppercase{\fetchmarking[book][][top] \fetchmarking[section][][top]}}\hfill]

\setuppagenumbering[location=]

\newwrite \referencegraph
\openout \referencegraph = referencegraph.txt\relax

\write\referencegraph{"source","target"}

\def\currentSection{none}
\def\writereference#1{\ignorespaces\write\referencegraph{"\currentSection","#1"}}

\def\inpropstr{pr.}
\def\inprop[#1]{\in{(\inpropstr}{)}[#1]\writereference{#1}}
\def\inpropL[#1]{\in{\inpropstr}{}[#1]\writereference{#1}}
\def\inpropN[#1]{\in{}{}[#1]\writereference{#1}}

\def\inpoststr{post.}
\def\inpost[#1]{\in{(\inpoststr}{)}[#1]\writereference{#1}}
\def\inpostL[#1]{\in{\inpoststr}{}[#1]\writereference{#1}}

\def\indefstr{def.}
\def\indef[#1]{\in{(\indefstr}{)}[#1]\writereference{#1}}
\def\indefL[#1]{\in{\indefstr}{}[#1]\writereference{#1}}

\def\inaxstr{ax.}
\def\inax[#1]{\in{(\inaxstr}{)}[#1]\writereference{#1}}
\def\inaxL[#1]{\in{\inaxstr}{}[#1]\writereference{#1}}

\def\plateName{plate}

\define[2]\bookTitle{\vbox{\vskip -\topspace\hskip0pt\hfill\hbox{\hskip-1pt\putLettrine{\plateName}\hskip-1pt}\hfill\hskip0pt\vskip 5\baselineskip~\hfill\hbox{\tfd #2}\hfill~}}

\definehead[intro][coupling=part,default=part,incrementnumber=no,placehead=yes]
\setuphead[intro][alternative=text,command=\bookTitle,header=empty,footer=none,before={},after={}]

\definehead[book][coupling=part,default=part,incrementnumber=yes,placehead=yes]
\setuphead[book][alternative=text,command=\bookTitle,header=empty,footer=none,before={},after={}]

\definecounter[book][way=bytext,numberconversion=R]
\define\startBook{\dodoubleempty\startbookalt}
\define\stopBook{\stopbook}
\def\startbookalt[#1][#2]{%
  \incrementcounter[book]%
  \startbook[ownnumber={\directconvertedcounter{book}\empty},incrementnumber=no,#1][#2]}

\definehead[supersection][coupling=chapter,default=chapter,incrementnumber=no,placehead=yes,page=no]
\define[2]\supersectionTitle{\midaligned{\tfb #2}}
\setuphead[supersection][command=\supersectionTitle,before={\blank[line]},after={\blank[line]}]

\definehead[silentSupersection][coupling=chapter,default=chapter,incrementnumber=no,placehead=yes,page=no]
\define[2]\noTitle{}
\setuphead[silentSupersection][alternative=text,command=\noTitle,before={},after={},inbetween={}]

\definehead[subproposition][subsection]
\define[2]\subpropositionTitle{\midaligned{#2}}
\setuphead[subproposition][command=\subpropositionTitle,before={\blank[line]},after={\blank[line]}]

%\definehighlight[emph][style=italic]
\def\emph#1{{\em #1}}

\setuplines[align=middle]

\setupindenting[yes,medium]

\setupinteraction[state=start,color=black,style=regular]

\setupmargindata[inouter][style=normal] % otherwise text in margins is set in defferent font

\def\marginNote#1{\inouter{~\hfill\framed[frame=off,location=top,width=4cm,align=flushleft,bodyfont=8pt]{#1}\hfill~}}

\def\figureInMargin{\dosingleempty\figureinmargin}
\def\figureinmargin[#1]#2{\ignorespaces
	\iffirstargument\ignorespaces
		\signalrightpage\inouter{~\doifrightpageelse{}{\hfill}\framed[frame=off,location=top]{#2}\doifrightpageelse{\hfill}{}~}\ignorespaces
	\else\ignorespaces
		\inouter{~\hfill\framed[frame=off,location=top]{#2}\hfill~}\ignorespaces
	\fi\ignorespaces
}

\def\drawCurrentPictureInMargin{\dosingleempty\drawcurrentpictureinmargin}
\def\drawcurrentpictureinmargin[#1]{
	\iffirstargument\ignorespaces
		\figureInMargin[#1]{\drawCurrentPicture}\ignorespaces
	\else\ignorespaces
		\figureInMargin{\drawCurrentPicture}\ignorespaces
	\fi\ignorespaces
}

\def\symb#1{\noindent{#1}\emspace}

% these colors  are duplicates of the colors in byrne.mp
\definecolor[byred][r=0.85,g=0.3,b=0.1]
\definecolor[byblue][r=0.15,g=0.35,b=0.6]
\definecolor[byyellow][r=0.95,g=0.7,b=0.1]
\definecolor[byblack][r=0,g=0,b=0]

\def\sepSpace{\vskip 0.75\baselineskip}

\definestartstop[CenterAlign][
before={\resetII\startalignment[middle]\setupwhitespace[big]},
after={\stopalignment\vskip 0.5\baselineskip}]

% http://tex.stackexchange.com/questions/326108/parallel-section-numberings-in-context

\unprotect

\def\setCurrentSection{\global\edef\currentSection{\currentstructurereference}}

\definehead[silentsection][section]
\define[2]\noTitle{}
\setuphead[silentsection][alternative=text,command=\noTitle,before={},after={},inbetween={}]

\definehead[sectiononlyname][section]
\define[2]\definitionTitle{\midaligned{#2}}
\setuphead[sectiononlyname][command=\definitionTitle,before={\blank[halfline]},after={\blank[halfline]}]

\definehead[sectiononlynumber][section]
\define[2]\definitionNumber{\midaligned{#1}}
\setuphead[sectiononlynumber][command=\definitionNumber,before={\blank[quarterline]},after={\blank[quarterline]}]

\def\preSectionCommand{}
\def\postSectionCommand{}

\definecounter[proposition][way=bypart]
\define\startProposition{\dodoubleempty\startproposition}
\define\stopProposition{\postSectionCommand\stopsilentsection}
\def\startproposition[#1][#2]{%
  \incrementcounter[proposition]%
  \startsilentsection[ownnumber={\directconvertedcounter{proposition}\empty},
    incrementnumber=no,#1][#2]\setCurrentSection\preSectionCommand}
    
\define\startVerboseProposition{\dodoubleempty\startverboseproposition}
\define\stopVerboseProposition{\postSectionCommand\stopsectiononlyname}
\def\startverboseproposition[#1][#2]{%
  \incrementcounter[proposition]%
  \startsectiononlyname[ownnumber={\directconvertedcounter{proposition}\empty},
    incrementnumber=no,#1][#2]\setCurrentSection\preSectionCommand}

\definecounter[propositionaz][way=bypart,numberconversion=Characters]
\define\startPropositionAZ{\dodoubleempty\startpropositionaz}
\define\stopPropositionAZ{\postSectionCommand\stopsilentsection}
\def\startpropositionaz[#1][#2]{%
  \incrementcounter[propositionaz]%
  \startsilentsection[ownnumber={\directconvertedcounter{propositionaz}\empty},
    incrementnumber=no,#1][#2]\setCurrentSection\preSectionCommand}

\definecounter[definition][way=bypart]
\define\startDefinition{\dodoubleempty\startdefinition}
\define\stopDefinition{\stopsectiononlyname}
\def\startdefinition[#1][#2]{%
  \incrementcounter[definition]%
  \startsectiononlyname[ownnumber={\directconvertedcounter{definition}\empty},
   incrementnumber=no,#1][#2]\setCurrentSection}

\definecounter[definitionaz][way=bypart,numberconversion=Characters]
\define\startDefinitionAZ{\dodoubleempty\startdefinitionaz}
\define\stopDefinitionAZ{\stopsectiononlyname}
\def\startdefinitionaz[#1][#2]{%
  \incrementcounter[definitionaz]%
  \startsectiononlyname[ownnumber={\directconvertedcounter{definitionaz}\empty},
   incrementnumber=no,#1][#2]\setCurrentSection}

\define\startDefinitionOnlyNumber{\dodoubleempty\startdefinitiononlynumber}
\define\stopDefinitionOnlyNumber{\stopsectiononlynumber}
\def\startdefinitiononlynumber[#1][#2]{%
  \incrementcounter[definition]%
  \startsectiononlynumber[ownnumber={\directconvertedcounter{definition}\empty},
   incrementnumber=no,#1][#2]\setCurrentSection}

\definecounter[axiom][way=bypart,numberconversion=R]
\define\startAxiomOnlyNumber{\dodoubleempty\startaxiomonlynumber}
\define\stopAxiomOnlyNumber{\stopsectiononlynumber}
\def\startaxiomonlynumber[#1][#2]{%
  \incrementcounter[axiom]%
  \startsectiononlynumber[ownnumber={\directconvertedcounter{axiom}\empty},
   incrementnumber=no,#1][#2]\setCurrentSection}

\definecounter[postulate][way=bypart,numberconversion=R]
\define\startPostulateOnlyNumber{\dodoubleempty\startpostulateonlynumber}
\define\stopPostulateOnlyNumber{\stopsectiononlynumber}
\def\startpostulateonlynumber[#1][#2]{%
  \incrementcounter[postulate]%
  \startsectiononlynumber[ownnumber={\directconvertedcounter{postulate}\empty},
   incrementnumber=no,#1][#2]\setCurrentSection}

\protect

\def\overText#1#2{\mathop{#1}\limits_{}^{\mbox{#2}}}
\def\underText#1#2{\mathop{#1}\limits_{\mbox{#2}}^{}}

\definedescription[symblist][location=hanging,headstyle=bold,style=normal,align=left]

% This section is about fonts

\emergencystretch = 3pt

\setupalign[hz,hanging]
	
\usesymbols[cc]
